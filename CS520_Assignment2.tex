%%%%%%%%%%%%%%%%%%%%%%%%%%%%%%%%%%%%%%%%%%%%%%%%%%%%%%%%%%%%%%%%%%%%%%%%%%%%%%%%%%%%
% Do not alter this block (unless you're familiar with LaTeX
\documentclass{article}
\usepackage[margin=1in]{geometry} 
\usepackage{amsmath,amsthm,amssymb,amsfonts, fancyhdr, color, comment, graphicx, environ}
\usepackage{xcolor}
\usepackage{mdframed}
\usepackage[shortlabels]{enumitem}
\usepackage{indentfirst}
\usepackage{hyperref}
\hypersetup{
    colorlinks=true,
    linkcolor=blue,
    filecolor=magenta,      
    urlcolor=blue,
}


\pagestyle{fancy}


\newenvironment{problem}[2][Problem]
    { \begin{mdframed}[backgroundcolor=gray!20] \textbf{#1 #2} \\}
    {  \end{mdframed}}

% Define solution environment
\newenvironment{solution}{\textbf{Solution}}

%%%%%%%%%%%%%%%%%%%%%%%%%%%%%%%%%%%%%%%%%%%%%
%Fill in the appropriate information below
\lhead{Michael Yen and Jay Patel}
\rhead{CS520} 
\chead{\textbf{Assignment 2}}
%%%%%%%%%%%%%%%%%%%%%%%%%%%%%%%%%%%%%%%%%%%%%


\begin{document}

\begin{problem}{1}
Consider the game search tree given in the figure below.
a. Apply MinMax (minimax) algorithm to the game tree. Write down the intermediate values at
the nodes. What is the best sequence of moves at the max node a (a path along the tree)? What
is the utility?
b. Apply MinMax with alpha-beta pruning in left-to-right order. Write down all nodes that are
visited in the order they are visited. Here, by “visited”, we mean the first time a node is seen by
the algorithm. Whenever truncation is applied, i.e., when part of the tree is skipped, provide the
node where this is happening and explain why the truncation is applied. In particular, write down
the values of $\alpha$ and $\beta$ at this point.
\end{problem}

\begin{solution} \\
\underline{Regular MinMax / part a} \\
a: 14 (c) \\
b: 12 (e) \\
c: 14 (j) \\
d: 9 (w) \\
e: 12 \\
f: 23 \\
g: 24 \\
h: 21 (n) \\
i: 26 (q) \\
j: 14 (s) \\
k: 10 (u) \\
l: 9 (w) \\
m: 12 \\ 
n: 21 \\
o: 25 \\
p: 2 \\
q: 26 \\
r: 11 \\
s: 14 \\ 
t: 4 \\
u: 10 \\
v: 8 \\
w: 9 \\
The best sequence of nodes at a is $a \rightarrow c \rightarrow j \rightarrow s$, leading to a utility of 14. \\
\underline{MinMax with alpha-beta pruning / part b} \\
a \\
b \\ 
e \\
f \\
g \\
c \\
h \\
m \\
n \\
i \\
o \\
At this point nodes p and q (w/ values 2 and 26) are pruned from the true because $[\alpha, \beta] = [12,21]$ and the value returned by the minimum function on node o was 25. Since $25 \ge 21$, the function finishes and returns 25. \\
j \\
r \\
s \\
d \\
k \\
t \\ 
u \\
At this point the node l and its children v and w (w/ values 8 and 9) are pruned from the tree because $[\alpha, \beta] = [14, \infty]$ and the value returned by the maximum function on node k was 10. Since $10 \le 14$, the function finishes and returns 10.
\end{solution}

\begin{solution}

\textbf{(a)}\\\relax

[a, 14]\relax

[b, 12]\relax

[c, 14]\relax

[d, 9]\relax

[h, 21]\relax

[j, 14]\relax

[k, 10]\relax

[l, 9]\relax


Best sequence of moves: $a \rightarrow c \rightarrow j \rightarrow s$.

This sequence gives the utility of 14.\\


\textbf{(b)}\\

format: [node, $\alpha$ value, $\beta$ value]\\

$[a, -\infty, \infty]$\relax

$[b, -\infty, \infty]$\relax

$[e, -\infty, 12]$\relax

$[f, -\infty, 12]$\relax

$[g, 12, 12]$\relax

So, value at b = 12\relax

$[c, -\infty, \infty]$\relax

$[h, -\infty, \infty]$\relax

$[m, ,12, \infty]$\relax

$[n, 21, \infty]$\relax

So, value at h = 21\relax

$[i, -\infty, \infty]$\relax

$[o, 25, \infty]$\relax

$[p, 25, \infty]$\relax

$[q, 26, \infty]$\relax

So, value at i = 26\relax

$[j, -\infty, \infty]$\relax

$[r, 11, \infty]$\relax

$[s, 14, \infty]$\relax

So, value at j = 14\relax

$[h, -\infty, 21]$\relax

$[i, -\infty, 21]$\relax

$[j, -\infty, 14]$\relax

So, value at c = 14\relax

$[d, -\infty, \infty]$\relax

$[k, -\infty, \infty]$\relax

$[t, 4, \infty]$\relax

$[u, 10, \infty]$\relax

So, value at k = 10\relax

Since d is a minimizer node, it will not let maximizer get a value higher than 10 and maximizer can get a higher value by following other routes. So, we can prune the right subtree of d which includes l, v and w nodes.

$[b, 12, 12]$\relax

$[c, 14, 12]$\relax

$[d, 14, 12]$\relax

So, value at a = 14\\
\end{solution}

\begin{problem}{2}
Game in normal form. Consider the payoff matrix given in the figure. For each entry,
the first number is P1’s payoff and the second number if P2’s payoff.
a. Is there a dominant strategy (from 1, 2, 3) for player 1? And player 2 (from a, b)? If the answer
is yes for either player, provide the corresponding dominant strategy or strategies. Justify your answer.
\end{problem}
\begin{solution}

There is a dominant strategy for player 1, which is to always choose the third action (3). Assume player 2 chooses (a). If player 1 chooses (1), they get a payoff of 1; if they choose (2), they get a payoff of 6;  if they choose (3), they get a payoff of 6. So (1) is eliminated from being dominant already. Now assume player 2 chooses (b). If player 1 chooses (2), they get a payoff of 3; if player 1 chooses (3) they get a payoff of 4. So (3) will always give a higher payoff than any of the other choices. \\

Player 2 does not have a dominant strategy. Assume player 1 chooses (1). If player 2 chooses (a) they get a payoff of 2; if player 2 chooses (b) they get a payoff of 4. So in that case player 2 should choose (b). Now assume player 1 chooses (2). If player 2 chooses (a) they get a payoff of 3; if player 2 chooses (b) they get a payoff of 1. In this case player 2 should choose (a). Since we already have two cases where player 2 should choose a different action, there is no dominant strategy. \\

One Nash equilibrium is at (3, b). Player 1 has nothing to gain by moving from (3, b) to either (1, b) or (2, b); their best payoff is still 4. Player 2 has nothing to gain from moving from (3, b) to (3, a), since that would lower their payoff from 7 to 3. Thus (3, b) is a Nash equilibrium. \\

Another Nash equilibrium is at (2, a). Player 1 cannot gain anything by moving from (2, a) to either (1, a) or (3, a); their best payoffi s 6. Likewise, Player 2 cannot gain anything by moving from (2, a) to (2, b), which would lower their payoff from 3 to 1. Thus (2, a) is a Nash equilibrium. \\

The last Nash equilibrium is at (1, b). Player 1 does not gain by moving from (1, b) to either (3, b) or (2, b). Similarly, Player 2 does not gain by moving from (1, b) to (1, a). So (1, b) is a Nash equilibrium. \\

Thus, there are 3 Nash equilibriums: (3, b), (2, a), and (1, b).
\end{solution}
\begin{problem}{3}
You are to manually run the MCTS algorithm for the navigation example covered in class for 10
iterations. For this, you will need simulation results for
nodes at tree depth up to 2, which is provided in the table
below. For node selection, you may need to pick among
four actions. The directions for the 10 rounds are preselected as N, W, S, E, W, N, E, N, E, S. If an action leads
to a node that is already simulated, pick the next node in
the counterclockwise direction for which a simulation has
not been performed.
For the answer, compute for each iteration which node is
selected for expansion, and update the reward and the number of visits at nodes after each iteration. At the end, indicate which action is predicted for the
next iteration at the root node.
\end{problem}
\end{document}